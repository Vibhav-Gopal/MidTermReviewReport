\chapter{Chapter 2}
\graphicspath{{Chapter2/}}

\section{Literature Survey}
% \subsection{Subsection heading}
% Type content here. This is how you cite a reference \cite{journal}.
% \begin{figure}[h]
% 	\centering
% 	\includegraphics[scale=0.3]{logo}
% 	\caption{Institute Logo}
% \end{figure}


\begin{itemize}

	\item Water Management and Policy Papers: Looked into studies and reports on water management policies, rules, and tactics for urban areas that have been released by governmental and non-governmental organizations. Seek information about the difficulties urban households have in controlling their water use.
	\item Technology Innovations and Solutions: Read up on the literature on technological advancements in water conservation and usage monitoring. Research on mobile apps, IoT gadgets, smart water meters, and other tools to assist users in monitoring and controlling their water use is included in this.
	\item Examined case studies and best practices from towns and cities that have successfully implemented water management initiatives. Seek information about the tactics, innovations, and laws that have been successful in encouraging water conservation and lowering usage in urban homes.

\end{itemize}

\section{Gaps in existings solutions}
\begin{itemize}
	\item Accessibility and affordability: Many of the water usage tracking solutions currently on the market may be too costly or out of reach for some urban households, especially those with tight budgets. Technologies that are more accessible and affordable and can be used by people from a variety of socioeconomic backgrounds are needed.
	\item Integration with Local Infrastructure: Water management and distribution systems in India differ greatly amongst cities and regions. Current solutions could be less practical or less effective to use if they don't work well with the local infrastructure, such as utility networks or water meters. Solutions that are flexible enough to adjust to India's varied infrastructure environment are required.
	\item Tailored for Regional Needs: Different parts of India have different water-related cultural norms and usage patterns. It's possible that current solutions aren't sufficiently tailored to these regional demands and preferences. Solutions that can be modified or tailored to meet the unique needs of various Indian communities and regions are needed.
	\item Government Support and Policy Alignment: Policies and rules set forth by the government have a significant impact on how water is managed in India. Current solutions might not comply with current laws or regulations, which would make it difficult for them to be adopted and scaled up. For solutions to be widely adopted and have an impact, they must be in line with government priorities and win the support of policymakers.
	
\end{itemize}

