\chapter{Chapter 4}
\graphicspath{{Chapter4/}}
\section{Deliverables}
\begin{itemize}
	\item Water Usage awareness
	\item Water Quality Monitoring implementation
	\item Intuitive and easy installation and interface
	\item Affordable solution compared to similar products
\end{itemize}

\section{Roles and Responsibilites}
\begin{enumerate}
	\item Vibhav\begin{itemize}
		\item CAD model ideation
		\item Design Ideation
		\item Electronics Design
		\item Technical Research/Review
	\end{itemize}
	\item Praneeth\begin{itemize}
		\item CAD model ideation	
			\item Design Ideation
		\item Electronics Design
		\item Market Research/Review
	\end{itemize}
	\item Vikyath\begin{itemize}
		\item CAD model ideation	
			\item Design Ideation
		\item Comp. Sci. Design
		\item Market Research/Review
		\item Time Planning and Progress Checks
	\end{itemize}
	\item Krishna Sahith\begin{itemize}
		\item CAD model ideation	
			\item Design Ideation
		\item Comp. Sci. Design
		\item Design Realisation/ BOM generation
	\end{itemize}

\end{enumerate}
\section{Future Work/ Conclusion}
\begin{itemize}
	\item Improved Predictive Analytics: Expand on previous work and incorporate outside variables like population expansion and weather patterns to predict future trends in water usage.
	\item Community-Based Initiatives: Launch community-based water conservation initiatives, such as awareness campaigns, incentive programs, and community-led conservation projects, in partnership with your local government, non-profit organizations, and communities.
	\item Scalability and Global Reach: Expand the solution to include rural communities, businesses, and governmental organizations in addition to Indian urban households. Examine potential for international growth and collaboration with foreign partners.
\end{itemize}
In conclusion, there are a number of options available for monitoring water consumption in households; however, the current situation highlights a number of shortcomings and difficulties, particularly in Indian urban households. The inadequacies of current solutions in terms of affordability, accessibility, and granular data analytics make them less effective in meeting the particular requirements of these communities.\\
Since many solutions are either too costly or out of reach for a sizable portion of the population—particularly those living in low-income urban areas—accessibility continues to be a major concern. 
Furthermore, there are difficulties in integrating with the local infrastructure and being compatible with the current water management systems, especially in places like India that have diverse infrastructure landscapes. A lot of solutions might not work well with neighborhood water meters or utility networks, which would lessen their usefulness and efficacy.\\
Moreover, users may be discouraged from implementing some solutions due to worries about security and privacy, especially when those solutions call for the collection of personal data. Users might be reluctant to give these platforms access to their usage data if they don't have strong data privacy safeguards in place.\\
In summary, while current household water usage tracking systems provide insightful data and useful tools, more easily accessible, reasonably priced, and finely tuned solutions that are specifically suited to the requirements of Indian urban households are obviously needed. We can enable users to more effectively monitor their water consumption, encourage conservation efforts, and support sustainable water management practices by addressing these gaps and challenges.
