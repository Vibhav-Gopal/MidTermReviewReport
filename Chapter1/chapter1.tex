\chapter{Chapter 1}
\graphicspath{{Chapter1/}}
\section{Team Members}
\begin{enumerate}
	\item Mudigonda Vibhavgopal Lakshmi Narasimha - EC21B1027
	\item Tadepalli Praneeth - EC21B1018
	\item Bonagiri Krishna Sahith - CS21B1048
	\item Amaraneni Vikyath -  CS21B1045
\end{enumerate}
\section{Problem Statement}
% \subsection{Subsection heading}
% Type content here. This is how you cite a reference \cite{journal}.
% \begin{figure}[h]
% 	\centering
% 	\includegraphics[scale=0.3]{logo}
% 	\caption{Institute Logo}
% \end{figure}
An individual seeks a solution to track their daily household water usage statistics accurately.Effective usage of water, by tracking and monitoring the consumption patterns.
"I want a water usage tracking solution for urban households who struggle to accurately monitor their daily consumption patterns."


\section{Who are we addressing}
The problem statement is addressed to individuals or entities involved in developing solutions for water conservation and management, such as:
\begin{itemize}

	\item Environmental Organisations
	\item Technological Innovators
	\item Urban Planners
	\item Government Agencies
	\item Water Utility Companies
	\item Research Institutions
	\item Sustainable Development advocates


\end{itemize}
These stakeholders are encouraged to recognize the significance of addressing the challenges faced by urban households in monitoring their water usage accurately and to collaborate in developing effective solutions to tackle this issue.

\section{Why}
This problem statement is being addressed for a number of crucial reasons:
Resource conservation is important because water is a limited resource that is necessary for life and must be used wisely to protect it for future generations. We help preserve this valuable resource by tackling the difficulty of precisely tracking water usage.
\begin{itemize}
	\item Environmental Impact: Using too much water can harm nearby ecosystems, pollute water supplies, and make shortages worse in areas where there is already a shortage. We lessen these environmental effects by encouraging precise monitoring and thoughtful consumption.
	\item Sustainability: Targets for sustainable development, such as those pertaining to public health, environmental preservation, and economic stability, depend on sustainable water management. Solving this problem statement is consistent with more general sustainability goals.
	\item Empowerment of the Consumer: Giving households the means to keep an eye on their water usage gives people the power to decide on their consumption patterns with knowledge. Changes in behavior that decrease waste and encourage conservation may result from this awareness.
	\item Social Responsibility: It is our duty as environmental stewards to address problems pertaining to sustainable usage and water conservation. We are doing our part to encourage responsible resource management by taking on this problem statement.
	
\end{itemize}
These stakeholders are encouraged to recognize the significance of addressing the challenges faced by urban households in monitoring their water usage accurately and to collaborate in developing effective solutions to tackle this issue.
